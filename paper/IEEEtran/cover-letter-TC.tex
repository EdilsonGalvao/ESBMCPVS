\documentclass[11pt]{article}

\usepackage[english]{babel}
\usepackage[latin1]{inputenc}
\usepackage{graphicx}
\usepackage{amssymb}
\usepackage{url}

%\usepackage{mathtools}
%\mathtoolsset{showonlyrefs}

\usepackage{color}
\usepackage{natbib}
% \usepackage[style=authoryear-comp,dashed=false]{biblatex}
% \bibliography{piecewise_lyapunov}

\textwidth 15cm
\setlength{\textheight}{1.1\textheight}
\newcommand\hi{\hspace*{\parindent}}
\newcommand\vi{\vspace{\baselineskip}}
\newcommand\lac{{\mbox{{\Huge\bf L}\hspace{-0.65em}
\raisebox{-1.2ex}{\Huge\bf A}\hspace{-1.1em}
\raisebox{-0.6ex}{\Huge\bf C~}}}}


\newcommand{\Real}{\mathbb{R}}



\pagestyle{empty}

\begin{document}

\hoffset -1.5cm \voffset .5cm


\sf

\vspace*{-2.5cm} \hspace*{-0.1cm} {\small
{\mbox{\begin{minipage}{1.5cm}
\centerline{\includegraphics[width=2cm]{manchester.jpg}}
\end{minipage}}} \hspace*{0.1cm}
\begin{minipage}{11.0cm}{\sf \large \textbf{UNIVERSITY OF MANCHESTER}}\\
{\sc School of Computer Science}
\end{minipage}
} \vspace*{0mm}

\hspace*{-.7cm} {\rule[-1ex]{15cm}{0.03cm}}

\begin{flushright}
\begin{minipage}{7.0cm}\small
{\bf Please Reply to:}\\
Dr. Cordeiro\\
University of Manchester\\
Department of Computer Science\\
Kilburn Building, Manchester M13 9PL\\
{\em e-mail: lucas.cordeiro@manchester.ac.uk}
\end{minipage}
\end{flushright}

\vi
\hspace*{\fill}{\small Manchester, \today.}
\vi

\begin{flushleft}
	Amir G. Aghdam \\
	Editor-in-Chief \\
	Concordia University \\
  Canada \\
	IEEE Systems Journal
    \end{flushleft}
 
\indent Dear Amir
\vi

Please find enclosed our manuscript entitled ``Multi-core synthesis and maximum satisfiability for optimal sizing of solar photovoltaic systems'' by Edilson Galvao, Alessandro Trindade, and Lucas Cordeiro, which we would like to submit for publication in the journal of IEEE Systems Journal.

The present work describes and evaluates an automated synthesis methodology to obtain the optimal sizing of photovoltaic systems. It employs model checking, specific solvers, and multi-core features to synthesize photovoltaic (PV) systems used in rural areas or where grid extension is unfeasible.

The sizing of PV systems is usually performed by hand or with commercial specialized simulation-optimization tools, which are well known in the market. However, the exploration of all design space states is impossible with those tools, and some flaws, which can be originated from the design phase, reaching the field after the PV deployment. This can cause dissatisfaction to the PV system owners and conclude that intermittent renewable systems are not suitable.
   
Our study's experimental results from seven case studies showed that only the automated synthesis could find the most meaningful and detailed equipment information for PV systems in a comparative evaluation of tools. Finally, based on the fact that only since 2019 papers deal with automated synthesis applied to PV systems, with notable results but with low performance, our study is unique and with outstanding results.

The submitted paper is a substantially extended version of our
manuscript published at the VSTTE 2020 (12th Working Conference on Verified Software: Theories, Tools, and Experiments). In a ``Summary of Changes'' letter, submitted together with this manuscript, the authors summarize the new material.

\vi
Thank you for your time.
\vi
\indent
~~~~~~~~~~~~~~~~~~~~~~~~~~~Sincerely,\\



\begin{quote}
\begin{quote}
\begin{flushright}


%\vfill

Lucas C. Cordeiro~~~~~~~
\end{flushright}
\end{quote}
\end{quote}


 
 

\end{document} 
