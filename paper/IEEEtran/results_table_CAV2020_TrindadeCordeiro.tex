% This is samplepaper.tex, a sample chapter demonstrating the
% LLNCS macro package for Springer Computer Science proceedings;
% Version 2.20 of 2017/10/04
%
\documentclass[runningheads]{llncs}
%
\usepackage{xcolor}
\usepackage{graphicx}
\usepackage{makecell}
\usepackage{textcomp}
%\usepackage{amsmath,amssymb,amsfonts}
\usepackage{algorithm,algorithmic}
\usepackage{geometry}
\geometry{
  a4paper,         % or letterpaper
  textwidth=16cm,  % llncs has 12.2cm
  textheight=25cm % llncs has 19.3cm
%  heightrounded,   % integer number of lines
%  hratio=1:1,      % horizontally centered
%  vratio=2:3,      % not vertically centered
}

\renewcommand{\baselinestretch}{0.97}

\begin{document}
%
%\title{Automated Formal Synthesis of Optimal Sizing for Stand-alone Solar Photovoltaic Systems\thanks{Supported by Newton Fund (ref. 261881580) and FAPEAM (Amazonas State Foundation for Research Support, calls 009/2017 and PROTI Pesquisa 2018).}}
\title{Synthesis of Solar Photovoltaic Systems: Optimal Sizing Comparison}
%
%\titlerunning{Abbreviated paper title}
% If the paper title is too long for the running head, you can set
% an abbreviated paper title here
%
\author{Alessandro Trindade\inst{1} \and Lucas Cordeiro\inst{2}} %\orcidID{0000-0001-8262-2919}  \orcidID{0000-0002-6235-4272}
%
\authorrunning{A. Trindade and L. Cordeiro}
% First names are abbreviated in the running head.
% If there are more than two authors, 'et al.' is used.
%
\institute{Federal University of Amazonas, Brazil \email{alessandrotrindade@ufam.edu.br} \and
University of Manchester, UK \email{lucas.cordeiro@manchester.ac.uk}}
\maketitle              % typeset the header of the contribution

\begin{table}[!h]
\caption{CAV 2020 case studies and results: optimization of PV systems.}\label{tab1}
\begin{tabular}{|c|c|c|c|c|}
\hline
\hline
Tools & \makecell{CPAchecker 1.8\\(MathSAT 5.5.3)}& \makecell{CBMC 5.11\\(MiniSat 2.2.1)} & \makecell {ESBMC 6.0.0\\(Boolector 3.0.1)} & HOMER Pro 3.13.1\\
\hline
\hline
Specification & Result & Result & Result & Result \\
\hline
\makecell{\textbf{Case Study 1}\\Peak:342W\\Surge:342W \\E:3,900Wh/day\\Autonomy:48h} & \makecell{SAT (172.03 min) \\NTP:1$\times$340W (1S)\\NBT:8$\times$105Ah (2S-4P)\\Controller 15A/75V\\Inverter 700W/48V\\LCC: US\$ 7,790.53} & OM & TO & \makecell{(Time: 0.33 min)\\2.53 kW of PV\\NBT:12$\times$83.4Ah (2S-6P)\\0.351kW inverter\\LCC: US\$ 7,808.04}\\
\hline
\makecell{\textbf{Case Study 2}\\Peak:814W\\Surge:980W\\E:4,880Wh/day\\Autonomy:48h} & \makecell {SAT (228.7 min) \\NTP:2$\times$330W (2S)\\NBT:10$\times$105Ah (2S-5P)\\Controller 20A/100V DC\\Inverter 1,200W/24V \\LCC: US\$ 8,335.90} & OM & TO & \makecell{(Time: 0.18 min)\\3.71 kW of PV\\NBT:20$\times$83.4Ah (2S-10P)\\0.817kW inverter\\LCC: US\$ 12,861.75} \\
\hline
\makecell{\textbf{Case Study 3}\\Peak:815W\\Surge:980W\\E:4,880Wh/day\\Autonomy:12h} & \makecell {SAT (166.13 min) \\NTP:4$\times$150W (4S)\\NBT:4$\times$80Ah (2S-2P)\\Controller 15A/100V DC\\Inverter 1,200W/24V \\LCC: US\$ 7,306.27} & OM & TO & Not possible \\
\hline
\makecell{\textbf{Case Study 4}\\Peak:253W\\Surge:722W\\E:3,600Wh/day\\Autonomy:48h} &  \makecell {SAT (143.71 min) \\NTP:4$\times$150W (4S)\\NBT:10$\times$80Ah (2S-5P)\\Controller 15A/75V\\Inverter 750W/24V \\LCC: US\$ 7,816.31} & OM & TO & \makecell{(Time: 0.23 min)\\2.42 kW of PV\\NBT:12$\times$83.4Ah (2S-6P)\\0.254kW inverter\\LCC: US\$ 7,677.95}\\
\hline
\makecell{\textbf{Case Study 5}\\Peak:263W\\Surge:732W\\E:2,500Wh/day\\Autonomy:48h} &  \makecell {SAT (134.93 min) \\NTP:1$\times$340W (1S)\\NBT:6$\times$105Ah (2S-3P)\\Controller 15A/75V\\Inverter 400W/24V \\LCC: US\$ 7,252.14} & OM & TO & \makecell{(Time: 0.18 min)\\1.59 kW of PV\\NBT:10$\times$83.4Ah (2S-5P)\\0.268kW inverter\\LCC: US\$ 6,175.57} \\
\hline
\makecell{\textbf{Case Study 6}\\Peak:322W\\Surge:896W\\E:4,300Wh/day\\Autonomy:48h} &  \makecell {SAT (235.75 min) \\NTP:2$\times$200W (2S)\\NBT:10$\times$105Ah (2S-5P)\\Controller 15A/75V\\Inverter 400W/24V \\LCC: US\$ 8,287.23} & OM & TO & \makecell{(Time: 0.22 min)\\3.15 kW of PV\\NBT:14$\times$83.4Ah (2S-7P)\\0.328kW inverter\\LCC: US\$ 9,112.45} \\
\hline
\makecell{\textbf{Case Study 7}\\Peak:1,586W\\Surge:2,900W\\E:14,000Wh/day\\Autonomy:48h} & TO & OM & TO & \makecell{(Time: 0.20 min)\\12.5 kW of PV\\NBT:66$\times$83.4Ah (2S-33P)\\1.60kW inverter\\LCC: US\$ 41,878.11} \\
\hline
\hline
\end{tabular}
\\Caption: OM= out of memory; TO= timeout; E= energy, NTP= total number of panels, NBT= total number of batteries, LCC= life cicle cost.
\end{table}

\end{document}